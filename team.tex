\section{The Company/Team}
\label{sec:team}

%Recommended length: 1-3 pages.
%\begin{itemize}
%\item Describe the company founders or key participants in this proposed project. What level of effort will these persons devote to the proposed Phase I activities? How does the background and experience of the team enhance the credibility of the effort; have they previously taken similar products/services to market?
%\item Describe your vision for the company and the company's expected impact over the next five years.
%\item If the company has existing operations, describe how the proposed effort would fit into these activities. Describe the revenue history, if any, for the past three years. Include government funding and private investment in this discussion.
%\item Will you have consultants or subawardees working on this project? If so, what is their expertise, affiliation, and contribution to the project?
%\end{itemize}
%
%\noindent
%--End of instructions--

Organized in 1991, BECS is privately held and employs approximately 100
people. It has outgrown its current
manufacturing facility of 24,000 sq.~ft.\ so is in the process of
renovating and moving into a new 42,000 sq.~ft.\ facility.
Its intellectual property is protected with 12 issued patents and
2 patents pending.

The design engineering team at BECS is comprised of 10 individuals, of
which Todd Steinbrueck is the software team lead (who will serve as
PI on the SBIR grant) and Roger Chamberlain is the VP Engineering (who
will serve as senior personnel on the SBIR grant).\footnote{Roger Chamberlain
also is Professor of Computer Science and Engineering at Washington
University in St.~Louis. While bringing over 25 years of research
experience to bear on the problem at hand, Dr.~Chamberlain will be
serving in his capacity as VP Engineering at BECS and will maintain
strict compliance with the conflict-of-commitment rules at the the university.}
Additional members of the design engineering team will contribute to
the effort as needed. Mr.~Steinbrueck and Dr.~Chamberlain will each
commit 1~month of time to the project, and BECS will dedicate an additional
one FTE of software engineering effort (as articulated in the budget).

This team has extensive experience in control equipment design and
implementation, including custom implementation of both the hardware and
the software. Presently, BECS designs and manufactures equipment that maintains
proper water chemistry (the focus of this proposal), monitors and controls
refrigeration in commercial settings (e.g., valve controllers that adapt
automatically to different refrigerants), ensures that chickens and hogs
have food and water on the farm (e.g., grain bin content monitoring,
feed auger controls, environmental monitoring and control of animal houses),
and delivers CATV signals (e.g., RF amplifiers, line extenders).

In the aquatics market, BECS has set for itself a strategic goal of not
simply being an equipment provider (although it will continue in that
role), but rather being a company that is a strong partner in maintaining
optimal water chemistry. This notion implies a substantial increase
in service to owners/operators of control equipment.
Essentially, we are evolving into a new reality in which we are as much a
service company as a manufacturing company.

As described in Section~\ref{sec:background} below, BECS has supported
remote connectivity to its controllers for years.  What is new is that
rather than simply providing access to data (and expecting end users
to make use of it on their own), BECS is increasingly finding ways to
help the end users maintain proper water chemistry by being actively involved
in the water chemistry management process itself.
While initial connectivity solutions
merely supported pull semantics (i.e., users had to
manually connect to the controller to
see what was going on), current systems actively contact users when
conditions warrant (e.g., alarm conditions, chemical stocks depleted),
pushing information to maintenance personnel on demand.
Through the combination of functionality within the control equipment 
and data services provided via the cloud, BECS's vision is one in
which water chemistry control is simpler to achieve, more reliable,
and less resource intensive (both in terms of human resources and
chemical resources).

The existing business model is primarily based upon equipment sales.
Annual revenues for FY14 through FY16 are \$9.9 million, \$10.7 million,
and \$11.8 million, respectively
(with \$13 million predicted for FY17), across all the markets BECS
serves.\footnote{These revenue numbers are proprietary information.}
To sustain the level of service we aspire to deliver, there must be
a revenue stream to support the required effort.  Our vision here is
a subscription model, where end users pay a monthly fee for the
top tier service model.
This revenue stream is already a reality in the agricultural (Ag) market.
For a monthly fee, data from our Ag equipment is collected automatically
(by BECS), retained in the cloud, and made available to the appropriate
owner/operator (i.e., farmer).  Absent that monthly subscription,
owners/operators of the Ag equipment are free to collect their data
themselves, directly from the controller.
We anticipate the ability of BECS to provide privacy-preserving,
aggregated data access to aquatic control equipment to be well worth
the cost of a monthly subscription.

BECS has received government funding on two separate occasions
(see Section~\ref{sec:prior} for details), both of which were SBIR/STTR
grants through the National Institutes of Health.  We have not applied for
government funding since the completion of those grants.

We will collaborate on this SBIR project with Dr.~Alvitta Ottley,
Asst.~Professor of Computer Science and Engineering at Washington University
in St.~Louis, and a newly recruited graduate student to Dr.~Ottley's lab.
In this proposed agenda, Dr. Alvitta Ottley and her team will lead the design and integration of the visualization front-end of the system that will allow clients to interact with the data.
Dr. Ottley has built and evaluated a range of visualization tools and designs~\cite{brown2014finding, hakone2017proact,ottley2015personality,peck2013using}. 
Relevant to the proposed work, her prior research has focused on designing visualizations to decision support~\cite{hakone2017proact,ottley2012visually,ottley2016improving} for non-experts. 
Her work has also made significant advancement toward evaluating the effectiveness of visualization designs~\cite{peck2013using,ziemkiewicz2013visualization}, and understanding how users interactions with visualization tools~\cite{brown2014finding,ottley2015personality}.  
Dr.~Ottley will commit 1 month of effort towards the project, and the graduate
student will be full-time for the full period (one year).

We will consult on the project with Dr.~Liyue Fan, Asst.~Professor
of Information Security and Digital Forensics, University at Albany (SUNY).
Dr.~Fan's research interests are in data privacy, spatiotemporal data
analysis, and database applications.  Her research is one of two
currently viable approaches to differentially private time series data,
and is the one we plan to focus on in this project.
Dr.~Fan will commit 2~weeks of effort towards the project.

Together the proposal team has the necessary knowledge and experience to undertake the proposed work. 
More specific information regarding coordination and collaboration plans is
provided in Section~\ref{sec:plan}.
