\section{Conclusion}
\label{sec:conclude}

It is clear that privacy of IoT-derived data is currently a challenge,
and differential privacy theory has the potential to put privacy practice
on a sound theoretical footing.  What still remains to be seen is whether
or not differential privacy theory is ``ready for prime time''
in particular use cases (ours being water chemistry).

Specifically, can the current state of the theory handle all of the
practical considerations that much be resolved for it to be effective
in the real world.
\begin{itemize}
\item Can we choose appropriate parameter values so as to achieve
sufficient privacy and maintain data utility, specifically
within the context of water chemistry control?
\item Can we effectively communicate the anonymized query results to
users in visual forms that include the inherent uncertainty due to privacy?
\item Can we combine differential privacy theory with other privacy
mechanisms for a multi-tiered privacy infrastructure?
\item While ensuring that cross-organizational data are private, can we
exploit those data to improve the practice of water treatment control?
\end{itemize}
These are the questions we intend to address as part of this STTR project.
Success will enable benefits across many more markets than just aquatics,
as virtually all of the IoT space has issues that, if not identical to
those we face, are certainly comparable.
