\section{The Commercial Opportunity}
\label{sec:opportunity}

%Recommended length: 2-4 pages.
%\begin{itemize}
%\item Is there a broader societal need you are trying to address with this commercial opportunity? Please describe.
%\item Describe the market and addressable market for the innovation. Discuss the business economics and market drivers in the target industry.
%\item How has the market opportunity been validated? Describe your customers and your basic business model.
%\item Describe the competition. How do you expect the competitive landscape may change by the time your product/service enters the market?
%\item What are the key risks in bringing your innovation to market?
%\item Describe your commercialization approach. Discuss the potential economic benefits associated with your innovation, and provide estimates of the revenue potential, detailing your underlying assumptions.
%\item Describe the resources you expect will be needed to implement your commercialization approach.
%\item Describe your plan and expected timeline to secure these resources.
%\end{itemize}
%
%\noindent
%--End of instructions--

Data from IoT devices are growing at prolific rates, and the privacy concerns
associated with that data are growing equally fast. 
%Who owns the data, who
%should have access to the data, who controls access to the data, how can
%the data benefit the greater good, all of these are questions that currently
%do not have settled answers. 
 Who owns the data? Who
should have access to the data? Who controls access to the data? How can
the data benefit the greater good? All of these are questions that currently
do not have settled answers.  We are interested in pursuing a scenario
where data owners are willing to share their data to enable some greater
good, with some reasonable privacy assurances provided to those data owners.

The specific market that is the focus of this proposal is aquatics, or
water chemistry control. BECS currently provides control equipment in
the aquatics market, which includes a number of (mostly distinct)
market segments (e.g., pools, including municipal pools, commercial
pools, water parks, health clubs, animal habitats in zoos, residential pools;
drinking water treatment facilities; waste water treatment facilities;
fountains; building cooling towers; irrigation water; and
various commercial operations, including washing vegetables, fracking; etc.).
BECS also provides control equipment in the
agriculture and refrigeration markets (primarily livestock feeder controls
in agriculture and valve controls in refrigeration).
While the STTR-funded effort will focus on our aquatics control line;
if technically successful, we anticipate expanding it to agriculture
and refrigeration also.

BECS has recently introduced EZAnalytics\texttrademark{}, which enables
individual owners/operators of its aquatics equipment to collect data
logs and retain them in the cloud~\cite{ccgss17}.
These logs include sensor readings (pH, oxidation-reduction potential, free Cl
concentration, temperature, etc.), alarms (readings out of range, etc.),
user actions (e.g., set-point changes, etc.), and feed chemical inventories.
These data are quite valuable for compliance verification, diagnostics,
and the like. 
% Since the number of bodies of water administered by each
%organization is small, however, greater insights that can be revealed
%by aggregating across ownership boundaries are, as yet, still unavailable.
The number of bodies of water administered by each
organization is small, however, greater insights that can be revealed
by aggregating across ownership boundaries.
Our intention is to develop a set of privacy preserving mechanisms that allow
the aquatics community as a whole to learn a number of things from these
data:
\begin{itemize}
\item The effectiveness of control decisions.  How well do different approaches
maintain water chemistry?
\item How to recommend better control approaches. How well will control
approaches that are effective in one body of water transfer to another?
\item How to minimize chemical usage. Can we make water chemistry control
more cost effective by decreasing the consumables needed?
\item How to detect problems. Alarm conditions are problematic for multiple
reasons.  Keeping them from happening at all is almost always the best option.
\end{itemize}
The above knowledge brings direct value to the end users of BECS's equipment.

BECS sells controllers (and related equipment) through over 20
distributors that each have exclusive coverage in a geographic region
(typically one or more states).
International sales are supported by distributors in Canada,
the UK (for Europe), and Australia (for Asia and the Pacific rim),
while coverage in Latin America is provided by our distributor in the
Florida region.

For new product introductions, we typically follow an annual cycle,
centered around the fall sales meeting, which is attended by virtually
all of our distributors.  What we are hearing from our distributors
is that they appreciate the gradual transition from ``selling equipment''
to ``improving water chemistry as a service'' (see discussion in
Section~\ref{sec:team} and attached letters of support).
Currently, the service that BECS provides is entirely supported via
equipment sales.  However, we anticipate transitioning to include a
subscription model (as we have already done in the agriculture market).

BECS is seen as a leader in the aquatics market segments it actively pursues
(primarily pools, fountains) and is a fairly small player in other
aquatics market segments.  Our aspiration is to expand our presence
in additional market
segments, and we believe the proposed research will enable us to move
in that direction.

Probably the largest market risk we face is the development of a strong
subscription-based revenue stream.  Technical success in the proposed
research will undoubtedly be accompanied by an increase in equipment sales;
especially if it enables us to penetrate some of the other aquatics
market segments.  Probably more important, from a risk perspective,
is moving to a
market positioning based on a combination of equipment sales and
water quality service will depend upon the development of recurring
revenue for the service provided.

In the aquatics market, we will pursue a similar approach to that which
we have already initiated in agriculture. First, we simplify the process
for end users. Some of the ways we do this include:
(1)~make installation straightforward,
(2)~proactively deal with communications or control issues so end users are not
inconvenienced,
(3)~provide information to end users in clear and useful ways.
Note, this latter item is a significant motivator for the visualization
research that is a significant part of the proposed work.

Second, we ensure that the economic model for the customer is appropriate.
When we can make a convincing argument that the customer's cost savings
(e.g., in reduced consumables, reduced maintenance, etc.) is greater than
the cost of the subscription, the ROI is practically instantaneous.

In some aquatics market segments, market penetration is strongly a function
of cost of the equipment.  An example of this is building cooling towers.
A cooling tower controller is typically asked to perform simple water
chemistry control, and relinquish all higher-level decisions to a
building management system. The historical limitation of this approach
for BECS is that, since our revenue stream has been based exclusively
on equipment sales, there is limited opportunity for revenue growth.

Success with the proposed STTR project could change that situation
substantially.  The subscription-based revenue model could be quite
beneficial, in that inexpensive on-site equipment, tied to a building
management system, can provide time series data on the water chemistry.
The economic argument for subscribing to a privacy-preserving aggregated
data set is then based on reducing operating costs by some amount
greater than the subscription fee.
Our primary focus is water chemistry, which is quite distinct from 
that of traditional
building management systems, so we have a credible claim that the service
we are providing isn't already present.
While we are unlikely to quantitatively
be able to make this argument until sometime during Phase~II, this is
precisely the direction we hope to go.

While how well the new technology will enable us to penetrate new
market segments is difficult to predict, a simple computation can provide
an estimate of recurring revenue from a subscription model in the
market segments we currently serve.
Annually, BECS ships approximately 2000 control systems that are
eligible for participation in a subscription data service, and the
typical lifetime of a controller is on the order of 10 years.  This gives
a controller population (based on the currently served market segments) of
about 20,000.  If approximately half of those subscribed, and the subscription
fee were to be \$20/month, this yields an annual revenue of
$(20,000)\cdot(0.5)\cdot(\$20/\mbox{month})\cdot(12\mbox{ months/yr}) = \$2.4$~million.
Clearly, that would go up as controller sales increase (particularly with
new market segments), and would be less if the fraction subscribing
were lower.

To fully succeed in our vision, we must both deliver quality service
and effectively market and sell subscriptions to the services we
deliver.  The proposed STTR project will help substantially in achieving
the former (even though the initial development of EZAnalytics\texttrademark{}
has been internally funded), and the latter will depend upon traditional
marketing and sales efforts (which are already ongoing in the aquatics
market, funded via equipment sales).
