\section{The Innovation}
\label{sec:innovation}

%Recommended length: 1-3 pages.
%\begin{itemize}
%\item Briefly describe the innovation. At what stage of technical development is the innovation? (A more detailed description can be provided in the Technical Discussion and R\&D Plan, as described below).
%\item Describe the key technical challenges and risks in bringing the innovation to market. Which of these will be your focus in the proposed Phase I project?
%\item Describe the status of the intellectual property associated with this project and how you plan to protect it.
%\item NSF Lineage: Does your project have roots in non-SBIR/STTR NSF funding, either to the company or other organizations/institutions? If possible, please list the NSF award number(s) and division(s).
%\end{itemize}
%
%\noindent
%--End of instructions--

Dating back to the 1990s, BECS has always provided remote connectivity
to its aquatics equipment.  The current editions provide for both
security and ease-of-use~\cite{ccgss16,ccgss18}, yet the data model
is that owners/operators of the control equipment only have access to
data from their own controllers.
A number of benefits accrue from access to this ``owned-only'' data,
including:
\begin{enumerate}
\item verifying compliance to regulations for the appropriate
governmental authorities,
\item remote notification of anomalous or alarm conditions,
\item maintaining stocks of chemical consumables at appropriate levels, and
\item diagnosing and correcting many water chemistry problems without
having to be on-site.
\end{enumerate}

The innovation we wish to pursue in this work is to enable
the \emph{additional} benefits that can accrue from aggregating data
across organizational boundaries.
One of the lessons of the recent explosion in the utility of machine
learning techniques is that (at least to first order) they are more
effective when you give them more data.  This need for enough data is
also true for learning by humans as well.  When we are trying to discern
general properties of a system under study, having examples that
illuminate a wide range of the operating range is crucial.

In our particular focus area (water chemistry), individual operators are
typically responsible for somewhere between one and ten separate bodies
of water.  These numbers are simply insufficient to learn general truths
about water chemistry control, certainly not truths that we don't already
know from theory (e.g., add acid and the
pH goes down in a very predictable way).
The application-specific things we want to learn are more likely to
be empirically driven (e.g,. when the backwash is engaged to clean the
filter, does that tend to increase or decrease pH levels).  
A reasonable (if not perfect) analogy is the following, we are trying
to use data to learn the kinds of things commonly known by a 
veteran operator with years of experience.

The benefits of data that cross organizational boundaries are particularly
strong when the operational characteristics of different bodies of
water are distinct from one another.  A simple example of this is
volume, a 1000~gallon body of water acts differently than a 1~million
gallon body of water.  In a facility that has one of each (and this
is quite common), what an operator learns from one body of water often
isn't that useful for the other body of water.  On the other hand, if
he/she can learn from data collected about 1000~gallon systems from
many organizations, and similarly from data collected about 1~million
gallon systems from many organizations, what is learned is much more
likely to be of direct benefit.

There are two fundamental questions that must be solved for the
vision articulated above to become reality.
\begin{itemize}
\item[(1)] We must gather the
data under circumstances that allow its use for our intended purpose
(explicitly addressing the privacy concerns of the data owners), and
\item[(2)] we must use the aggregated data to learn lessons about water
treatment control.
\end{itemize}

This Phase I proposal will focus almost exclusively on question~(1);
specifically, how to effectively preserve the privacy of the owners
of the data so that they are willing to contribute it for community
use in the first place.
While we will touch on question~(2) during the Phase~I project, it will
be limited in scope, with the bulk of the work needed to address question~(2)
being deferred to Phase~II.

We have chosen to address question~(1) initially, followed by question~(2)
for the following reasons:
\begin{itemize}
\item Success with~(1) is generalizable to a much wider set of circumstances.
If we are ultimately unsuccessful with question~(2) in the field of water
chemistry, we can still export our Phase~I results to other markets.
\item The answers to question~(1) will impact the techniques we use in
investigating question~(2). Differential privacy (our preferred approach)
adds uncertainty to queries on the aggregated data. How much uncertainty, at
what expense in ``privacy budget,'' must be known before we can effectively
quantitatively evaluate what we can learn from the data.
\end{itemize}

Ultimately, we are working toward an integrated system that allows clients to explore this aggregated data.
Having access to the data is only half the solution. 
The data must also be presented in an intuitive manner to facilitate ease of exploration and ease of decision-making. 
To this end, we will employ state-of-the-art visualization techniques for communicating time series data to clients. 

During Phase~I, we will focus on the following two components of this
system:
\begin{enumerate}
\item Development of an infrastructure that aggregates water chemistry
data into the cloud, supports queries against this data, and responds to
these queries with added noise guided by differential privacy theory.
At various levels of privacy budget (more specifically,
$\epsilon$ and $\delta$), assess the resulting uncertainty inherent
in the query responses provided.
\item Exploration and evaluation of appropriate visualizations of data that have been
anonymized using differential privacy theory.  With a focus on end users,
the goal is to communicate both the information inherent in the data as well
as uncertainty in the data.
\end{enumerate}
We articulate how we will judge success (or failure) of the Phase~I effort
in Section~\ref{sec:research}.

Privacy theory has a rich history of NSF support, and
one of the strengths of our team is
transition of theory to practice, as evidenced by our work in
auto-calibration of capacitive level sensors~\cite{lc03}
and proximity detectors~\cite{prox},
deadlock detection and avoidance in streaming data computations~\cite{labcl10},
applying state-of-the-art security to IoT devices~\cite{ccgss16,ccgss18},
and developing flowing-water titration mechanisms that are effective
for recirculating systems~\cite{cesw18,cew18}.

In addition, we are cognizant of the fact that differential privacy
theory for collections of time series data is not fully complete.
We will, as required, develop new theory to enable an appropriate
balance between privacy and utility.  One of the members of our
team, Dr.~Yevgeniy Vorobeychik, has significant experience in
privacy-preserving data sharing with biomedical data, which has much
stronger dissemination contraints (HIPPA regulation)
than is the case for our data.

As indicated by the fact that 3 of the above 5 reduction to practice
examples are documented
via patents or patent applications, it is often the case that reducing
theory to practice results in inventive steps, and BECS regularly pursues
patent protection for technologies that it develops.  This project will
be no exception.
