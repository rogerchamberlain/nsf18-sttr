\centerline{\Large Practical Privacy in the Internet-of-Things}

\paragraph{Introduction}
\label{sec:intro}

A number of high-profile examples~\cite{bz06,bk07} have illustrated
the difficulty inherent in maintaining the privacy of individuals
whose data are shared, even when anonymized.
Differential privacy~\cite{dwork11,dr14} provides a formal
framework for ensuring bounds on data leaks about individuals; however,
there are a number of questions that remain about its use in practical
settings.  These include the choice of appropriate privacy budget,
$\epsilon$, $\delta$, and utility.

We propose to empirically evaluate the suitability of differential
privacy techniques in data collected from the Internet-of-Things (IoT).
\emph{Can we effectively parametrize the use of differential privacy techniques
(possibly in combination with other techniques, such as
$k$-anonymity~\cite{samarati01,sweeney02},
$l$-diversity~\cite{mkgv07}, etc.)
while enabling meaningful conclusions to be drawn from the data?}

Our Phase~I investigation will restrict its focus to the aquatics market,
and we will expand into agriculture and possibly refrigeration during
Phase~II. Each of these markets is one in which we currently provide
equipment.

\paragraph{Company and Team}
\label{sec:team}

BECS Technology, Inc., (BECS) is a small business that provides
control equipment in
a number of markets (agriculture, aquatics, refrigeration, etc.).
In the modern era, that implies communicating with the equipment via
networks, placing the controllers that BECS manufactures squarely in
the Internet-of-Things (IoT).
BECS takes the security of its systems seriously, providing multiple
layers of state-of-the-art security mechanisms on the
equipment that it manufactures, while at the same time supporting
ease-of-use on the part of the equipment owners and installers~\cite{ccgss16}.
Organized in 1991, BECS is privately held and employs approximately 100
people. It has outgrown its current
manufacturing facility of 24,000 sq.~ft.\ so is in the process of
renovating and moving into a new 42,000 sq.~ft.\ facility.

The design engineering team at BECS is comprised of 10 individuals, of
which Todd Steinbrueck is the software team lead (who will serve as
PI on the SBIR grant) and Roger Chamberlain is the VP Engineering (who
will serve as senior personnel on the SBIR grant).  Roger Chamberlain
also is Professor of Computer Science and Engineering at Washington
University in St.~Louis.
While bringing over 25 years of research experience to bear on the problem
at hand, Dr.~Chamberlain will be serving in his capacity as VP Engineering
at BECS and will maintain strict compliance with the conflict-of-commitment
rules at the the university.

\paragraph{Market Opportunity}
\label{sec:market}

BECS has recently introduced EZAnalytics\texttrademark{}, which enables
individual owners/operators of its aquatics equipment to collect data
logs and retain them in the cloud~\cite{ccgss17}.
These logs include sensor readings (pH, oxidation-reduction potential, free Cl
concentration, temperature, etc.), alarms (readings out of range, etc.),
user actions (e.g., set-point changes, etc.), and feed chemical inventories.
These data are quite valuable for compliance verification, diagnostics,
and the like.  Since the number of bodies of water administered by each
organization is small, however, greater insights that can be revealed
by aggregating across ownership boundaries are, as yet, still unavailable.
Our intention is to develop a set of privacy preserving mechanisms that allow
the aquatics community as a whole to learn a number of things from these
data.
\begin{itemize}
\item The effectiveness of control decisions.  How well do different approaches
maintain water chemistry?
\item How to recommend better control approaches. How well will control
approaches that are effective in one body of water transfer to another?
\item How to minimize chemical usage. Can we make water chemistry control
more cost effective by decreasing the consumables needed?
\item How to detect problems. Alarm conditions are problematic for multiple
reasons.  Keeping them from happening at all is almost always the best option.
\end{itemize}
The above knowledge brings direct value to the end users of BECS's equipment.

\paragraph{Core Technology and Innovation}
\label{sec:tech}

While the above knowledge will benefit the entire aquatics community, it relies
on owners/operators of the equipment sharing their private data with the
community.  For this to happen, it is essential that their privacy be ensured.

What we propose to investigate in this SBIR project is how to transition
the theory of differential privacy~\cite{dwork11,dr14} into effective
practice on IoT data.  Differential privacy provides strong guarantees
that no incremental harm comes to an individual (or, in our case, organization)
if they are willing to share their private data into a common database
maintained by a trusted curator. Statistical queries to that database
have added noise included in query responses so that privacy is preserved.

The theory is robust and quite strong.  What remains is transition into
practice, and there are a number of interesting questions that we intend
to address as part of this investigation.
\begin{itemize}
\item Differential privacy's theory depends upon appropriate tuning of a number
of parameters (e.g., $\epsilon$, $\delta$).  How the privacy budget impacts
utility is not well understood.  We will investigate this relationship empirically.
\item How do we communicate the privacy that is effectively provided by differential privacy theory to users?
\item How do we combine the use of differential privacy with alternative
approaches~\cite{ct13}, such as $k$-anonymity~\cite{samarati01,sweeney02} and
$l$-diversity~\cite{mkgv07}?
\end{itemize}
BECS will serve as the trusted curator of the database (it is already serving
that role, providing segregated access to owners/operators of their own data).
We will investigate how previous privacy-preserving machine learning
experience~\cite{acgmmtz16,ss15} translates into the space of IoT data.

\paragraph{Competition}
\label{sec:competition}

There certainly are other companies that provide control equipment in
each of the markets BECS serves, and providing remote communications
capabilities to that equipment is essential in the modern era.  As is
true for much of the IoT, however, BECS's competitors appear to be
late to the game in terms of ensuring strong security infrastructure.

We are unaware of any efforts towards enabling privacy-preserving
data analysis in the markets we serve.

\paragraph{Conclusions}
\label{sec:conclude}

The theory for how to provide privacy-preserving shared data is robust.
We at BECS Technology would like to apply that theory to the real world
of IoT data.  We will start with aquatics data, and generalize subsequently
to agricultural data and refrigeration data.
