\section{Broader Impacts}
\label{sec:broader}

Repeating a statement from Section~\ref{sec:opportunity},
we are interested in pursuing a scenario
where data owners are willing to share their data to enable some greater
good, with some reasable privacy assurances provided to those data owners.
This statement will be directly pursued in this project in the context of water
chemistry; however, it is a statement that is broadly applicable
across virtually all of the IoT space.
It's especially relevant to IoT in healthcare, where privacy concerns
are particularly acute.
BECS will seek to broaden it to BECS's other markets of agriculture
and refrigeration.
All of the data associated with NSF's Smart and Connected Communities
initiative has a similar set of privacy concerns.
In short, society can benefit to good answers to the questions we pose here.

At the graduate education level, this work will support two graduate
students at Washington Univ.~in St.~Louis.
The students will become versed in both visualization and
privacy theory, further strengthening their educational experience.

We will leverage a pair of existing university programs to help us
attract students from traditionally underrepresented groups.  The Olin
Fellowship Program (for women) and the Chancellor's Fellowship Program
(aimed at underrepresented minority students) have had a successful track
record of enabling individuals to pursue graduate study.  In our
experience, the most effective method for attracting students from
underrepresented groups is by personal contact with a suitable role
model.  To facilitate this, we regularly ask the appropriately
qualified individuals in our group to be actively involved in the
recruiting process.  This cohort currently includes both Dr.~Ottley
and one minority (African-American) student.
