\clearpage
\section*{Project Summary}

%Overview description (potential outcome(s) of the proposed activity
%in terms of a product, process, or service).
%
%--End of instructions--

Electronic control systems in the modern era are regularly connected
to the Internet, becoming endpoint nodes in the Internet-of-Things (IoT).
It is common for these control systems to log (over time) sensor measurements,
control decisions, anomalous or alarm conditions, parameter settings,
and the like; and these data can be helpful in verifying compliance to
relevant regulations, improving control function, diagnosing problems,
and maintaining appropriate stocks of consumables.

An individual organization, however, is limited to seeing only the data logs
of control systems that it directly owns. Aggregating data from multiple
systems (crossing organizational boundaries) can provide benefits to all
owners of similar equipment (e.g., better knowledge of the impact of
control decisions, diminished usage of consumables, prediction of future
problems). We propose to provide anonymized data sets for mining, utilizing
the strong guarantees made available by differential privacy techniques.

\paragraph{Key Words:} IoT, differential privacy

\paragraph{Subtopic Name:} IT4. Cybersecurity; Authentication; Privacy

\medskip

\paragraph{Intellectual Merit}
This Small Business Technology Transfer Research Phase I project will
investigate
the practical viability of providing differential privacy guarantees to
owners of data collected by IoT devices.  We will attempt to answer the
question, ``Can differential privacy theory be effectively reduced to
practice?'' for IoT data.

BECS Technology, Inc., (BECS) manufactures control equipment in a
number of markets.
This equipment is part of the Internet-of-Things, and BECS takes the
security and privacy of its equipment, and data collected from that
equipment, very seriously.
Our approach to data privacy has historically been one based strictly 
on ownership, the owner of the equipment is the owner of the data from
that equipment, and the owner is the only one who is allowed access to
that data.
This research will investigate the viability of relaxing that strict,
owner-based partitioning of the data.  More can be learned from aggregated
data sets, yet owners must be assured that by sharing their data the
benefits outweigh the risks.

In collaboration with Washington University in St.~Louis, we will develop an integrated system that aggregates clients' data to better harness the power of the IoT.  
In particular, this research agenda has two complementary veins. (1) 
Through empirical study,
we will assess the tradeoffs inherent in differentially private 
data science (i.e., what can be learned from the data, at what ``privacy
budget'').
This investigation will be carried out using aquatics data from
controllers manufactured by BECS and installed primarily in North America
(and extending around the world).
(2) We will explore visualization designs for displaying this aggregated data as well as methods for representing data uncertainty. 
We will conduct a series of empirical studies to compare existing techniques for representing time series data and uncertainty to non-experts. 
Together, the results of these two research trajectories will give us the means to combine data in a privacy-preserving manner and enable clients to better understand their data to support decision-making.  



\medskip

\paragraph{Commercial/Broader Impacts}
Providing practical and effective privacy guarantees for IoT-derived data
is potentially transformative, not only for the markets that BECS currently
serves, but across the entire range of devices in the IoT.
By encouraging the owners of data to be willing to share data while having
strong assurances their privacy will be preserved, more data will be
willingly shared, to the greater societal good.

Education -- This project will be carried out in collaboration with
Washington University in St. Louis, supporting two graduate students.
The students will be trained with an appropriate balance of expertise in human reasoning, state-of-the-art privacy theory and practice, preparing them to conduct interdisciplinary research.
We will utilize existing programs at the university (Chancellor's
Fellowship and Olin Fellowship) to help attract members of traditionally
underrepresented groups.
