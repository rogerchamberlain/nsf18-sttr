\section{Introduction}
\label{sec:intro}

%Elevator Pitch (no more than one page).
%\begin{itemize}
%\item The Customer. Describe the expected customer for the innovation. What customer needs or market pain points are you addressing?
%\item The Value Proposition. What are the benefits to the customer of your proposed innovation? What is the key differentiator of your company or technology? What is the potential societal value of your innovation?
%\item The Innovation: Succinctly describe your innovation. This section can contain proprietary information that could not be discussed in the Project Summary. What aspects are original, unusual, novel, disruptive, or transformative compared to the current state of the art?
%\end{itemize}
%
%\noindent
%--End of instructions--


The Internet of Things (IoT) is ushering in an era where significant numbers of devices that perform control functions (e.g., process control,
manufacturing, etc.) are connected via wired or wireless networks.
BECS Technology, Inc., (BECS) is a small business that manufactures controllers for a number of markets (agriculture, aquatics, refrigeration, etc.) and seeks to maximize the benefits to its customers of the data made available by this connectivity.
However, the current data sharing models we employ (like most companies) limits each individual organization to seeing only the data logs of control systems that it directly owns. 
This data model significantly impedes the potential of such connectivity.

The opportunities of aggregating data across organizational boundaries are substantial.
For instance, this can lead to better knowledge of the impact of control decisions and diminished usage of consumables.
We can begin to use machine learning techniques to improve control functionality and predict future problems. 
While these opportunities can and do provide real benefit to owners/operators of IoT-based control equipment, there are challenges related to security, privacy, and data communication that must be overcome.
Both Kumar and Patel~\cite{kp14} and
Vasilomanolakis et al.~\cite{vdlgw15} describe these challenges as being
purvasive across all the IoT.
%The security challenge is to prevent unauthorized access to equipment and data.
%We have previously described our approach to controller
%security~\cite{ccgss16,ccgss17b}, providing multiple layers of state-of-the-art
%security mechanisms, while at the same time supporting
%ease-of-use on the part of the equipment owners and installers.

We desire that owners of IoT-based data share those data with the broader community for the benefit of all, but this will only happen if we can assure owners that their data will be kept private.
This proposal seeks to explore the feasibility of cross-organizational
data sharing by addressing the challenge of data privacy.
Additionally, we explore techniques for communicating this aggregated data to clients to encourage better decision-making.  Some high-profile examples~\cite{bz06,bk07} have illustrated the difficulty inherent in maintaining the privacy of individuals whose data are shared, even when anonymized.
Differential privacy~\cite{dwork11,dr14} provides a formal framework for ensuring bounds on data leaks about individuals; however, some questions remain about its use in practical settings.

What we propose to investigate in this SBIR project is how to transition
the theory of differential privacy into useful practice on IoT data, and how to communicate such aggregated data to clients.  
Differential privacy provides strong guarantees that no incremental harm comes to an individual (or, in our case, organization) if they choose to share their private data with a common database maintained by a trusted curator. Statistical queries to that database have added noise included in query responses to preserve privacy.
This added noise introduces uncertainty to the data, making data communication a primary challenge. 

The privacy theory is robust and quite strong.  What remains is transition into
practice, and there are a number of interesting questions that we intend
to address as part of this investigation.
\begin{itemize}
\item Differential privacy's theory depends upon appropriate tuning of a number
of parameters (e.g., $\epsilon$, $\delta$).  How the privacy budget impacts
utility is not well understood.  We will investigate this relationship empirically.
\item How do we communicate to users both the query responses and the
inherent associated uncertainty due to differential privacy?
\item How do we combine the use of differential privacy with alternative
approaches~\cite{ct13}, such as $k$-anonymity~\cite{samarati01,sweeney02} and
$l$-diversity~\cite{mkgv07}?
\end{itemize}
BECS will serve as the trusted curator of the database (it is already serving
that role, providing segregated access to owners/operators of their own data).
We will investigate how previous privacy-preserving machine learning
experience~\cite{acgmmtz16,ss15} translates into the space of IoT data
and how visualization can effectively be used by
laypersons~\cite{ottley2012visually} in the context of anonymized data.
